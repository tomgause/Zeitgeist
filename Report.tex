% Options for packages loaded elsewhere
\PassOptionsToPackage{unicode}{hyperref}
\PassOptionsToPackage{hyphens}{url}
%
\documentclass[
]{article}
\usepackage{amsmath,amssymb}
\usepackage{lmodern}
\usepackage{iftex}
\ifPDFTeX
  \usepackage[T1]{fontenc}
  \usepackage[utf8]{inputenc}
  \usepackage{textcomp} % provide euro and other symbols
\else % if luatex or xetex
  \usepackage{unicode-math}
  \defaultfontfeatures{Scale=MatchLowercase}
  \defaultfontfeatures[\rmfamily]{Ligatures=TeX,Scale=1}
\fi
% Use upquote if available, for straight quotes in verbatim environments
\IfFileExists{upquote.sty}{\usepackage{upquote}}{}
\IfFileExists{microtype.sty}{% use microtype if available
  \usepackage[]{microtype}
  \UseMicrotypeSet[protrusion]{basicmath} % disable protrusion for tt fonts
}{}
\makeatletter
\@ifundefined{KOMAClassName}{% if non-KOMA class
  \IfFileExists{parskip.sty}{%
    \usepackage{parskip}
  }{% else
    \setlength{\parindent}{0pt}
    \setlength{\parskip}{6pt plus 2pt minus 1pt}}
}{% if KOMA class
  \KOMAoptions{parskip=half}}
\makeatother
\usepackage{xcolor}
\IfFileExists{xurl.sty}{\usepackage{xurl}}{} % add URL line breaks if available
\IfFileExists{bookmark.sty}{\usepackage{bookmark}}{\usepackage{hyperref}}
\hypersetup{
  pdftitle={Zeitgeist Analysis},
  pdfauthor={Tom Gause and Liam O'Brien},
  hidelinks,
  pdfcreator={LaTeX via pandoc}}
\urlstyle{same} % disable monospaced font for URLs
\usepackage[margin=1in]{geometry}
\usepackage{graphicx}
\makeatletter
\def\maxwidth{\ifdim\Gin@nat@width>\linewidth\linewidth\else\Gin@nat@width\fi}
\def\maxheight{\ifdim\Gin@nat@height>\textheight\textheight\else\Gin@nat@height\fi}
\makeatother
% Scale images if necessary, so that they will not overflow the page
% margins by default, and it is still possible to overwrite the defaults
% using explicit options in \includegraphics[width, height, ...]{}
\setkeys{Gin}{width=\maxwidth,height=\maxheight,keepaspectratio}
% Set default figure placement to htbp
\makeatletter
\def\fps@figure{htbp}
\makeatother
\setlength{\emergencystretch}{3em} % prevent overfull lines
\providecommand{\tightlist}{%
  \setlength{\itemsep}{0pt}\setlength{\parskip}{0pt}}
\setcounter{secnumdepth}{-\maxdimen} % remove section numbering
\usepackage{listings}
\usepackage{amsmath, amsthm, amssymb, amsfonts}
\usepackage{graphicx}
\ifLuaTeX
  \usepackage{selnolig}  % disable illegal ligatures
\fi

\title{Zeitgeist Analysis}
\author{Tom Gause and Liam O'Brien}
\date{May 20, 2022}

\begin{document}
\maketitle

\hypertarget{introduction}{%
\section{Introduction}\label{introduction}}

For the past four years at Middlebury College, a small team of students
who write for the Middlebury Campus, the school's student newspaper,
have designed a survey called the Zeitgeist. The intent of this survey
is to get a basic idea of student behavior and attitude on campus. The
survey is totally anonymous, and this lets the Zeitgeist team ask a
range of sensitive questions such as ``Have you ever broken the Honor
Code?'', ``Have you been a victim of sexual assault?'', and ``Are you
happy''? Each year, this survey, which takes 10-20 minutes to complete,
is sent to the entire student body. When the responses are collected,
the Middlebury Campus analyzes the data and releases an article
summarizing some of the significant results of the survey (note that
this article shares statistics only, the Zeitgeist team avoids any
opinion writing). The Zeitgeist project is the largest and most
comprehensive student census at Middlebury.\\
In Spring 2022, the Zeitgeist team released the Zeitgeist 4.0. They
collected 1134 responses of 2837 undergraduates--approximately 40\% of
the student population--and published a
\href{https://www.middleburycampus.com/article/2022/05/zeitgeist-4-0-2022}{summary article}.
For our final project, we chose to answer the following question: are
the Zietgeist 4.0 respondents representative of the Middlebury student
population?\\
Our null hypothesis is: The Zeitgeist is a random sample from the
Middlebury student population. To evaluate this hypothesis, we used 5
metrics--Class (class year), Gender (gender identification), Race
(ethnicity), Geographic Distribution (hometown), and Major
(academic)--and collected data for each of these metrics from the
Zeitgeist summary post referenced above. For our Middlebury College
population data, we collected responses from various administrators and
public datasets. See Methods, Data Collection for details.

\hypertarget{methods}{%
\section{Methods}\label{methods}}

As we are testing our hypothesis with a large number of metrics, we
pre-planned the following steps to avoid a type I error, i.e.~avoid
p-hacking. First, we decided to only run permutation tests. We used the
formula \(\alpha^* = 1-(1-\alpha)^{\frac{1}{k}}\) where
\(\alpha = 0.05\) and \(k = 5\) to determine the value
\(<\alpha^* = \simeq 0.0102\) at which the p-value is considered
statistically significant. Next, there is some variation in the features
of the Zeitgeist and Middlebury Administration college data. To do
analysis, we need to combine or remove some features from both dataset.
To avoid bias, these data mutations were made before running our first
phase of statistical tests. If the p-values from any of the Phase 1
metric tests came up significant, we would engage in a series of Phase 2
tests where we would alter the data assumptions made about the metrics
before Phase 1 and re-test against the hypothesis. This process will
help ensure any detected significance is not due to falsely made
assumptions.

For each of our five categories (gender, graduation year, race, region
of residence, and major), we perform a permutation test. In each, we
treat the Zeitgeist data like a sample from the administration data,
which represents the Middlebury population. Our null hypothesis \(H_0\)
is that the Zeitgeist data is a random sample from the administration
data. We use mean absolute error (MAE) as our test statistic for our
Zeitgeist data. \begin{equation*}
\text{MAE} = \frac{1}{n}\sum |\text{observed} - \text{expected}|.
\end{equation*} We are summing over each categorical variable, and \(n\)
is the number of categorical variables. The observed count is the number
of students in a category, and the expected count is the number of
students that would be in the category if the proportion of students in
the Zeitgeist sample was equal to the proportion in the population at
Middlebury---as provided by the administration data.

\hypertarget{data-collection}{%
\subsubsection{Data Collection}\label{data-collection}}

The Zeitgeist data was collected from
\href{https://www.middleburycampus.com/article/2022/05/zeitgeist-4-0-2022}{this article}.
It is unknown what data modification/cleaning was done before publishing
the article, so we treat this data as ground truth for Zeitgeist. As the
Zeitgeist survey was done during the spring semester in 2022, we aimed
to retrieve data from Middlebury during the spring of 2022 as well. The
data from Middlebury's administration was collected from the
\href{Undergraduate College | Middlebury}{\textit{Office of Assessment and Institutional Research}}.
Their \textit{Spring 2022 Enrollment} document provided us information
about the total enrollment at Middlebury by gender and by ethnicity. The
Director of Assessment and Institutional Research at Middlebury, Adela
Langrock, generously put together geographic data for us; it totals the
states of residence for the enrolled students in the spring of 2022. We
did not have the number of each graduating class enrolled in the spring
of 2022; instead, we multiplied the proportion of each class in the fall
of 2021 by the total enrollment for the spring of 2022. We are assuming
that the proportions of each graduating class is the same in the fall
and spring semesters. The enrollment by graduation year for the fall is
contained in the \textit{Fall 2021 Student Profile} on the Office of
Assessment and Institutional Research webpage linked above.

\hypertarget{data-modification-pre-phase-1}{%
\subsubsection{Data Modification Pre Phase
1}\label{data-modification-pre-phase-1}}

All of the questions in the Zeitgeist included an ``I prefer not to
answer'' option. This option was provided for numerous reasons--some
respondents may have felt that the options provided did not apply to
them, others may have been concerned that an adversary on the Zeitgeist
Team would discover their identity from the provided information and use
their data in a malicious manner. For the metrics we chose to test, less
than \(1%
\) of respondents selected the ``I prefer not to answer'' option. For
this reason, we have omitted this category from the data for all metrics
for the initial testing phase. When describing the options available to
respondents, we will omit the ``I prefer not to answer'' option.

After removing the ``I prefer not to answer'' data, the Class Year, and
Academic Major metrics matched one-to-one, so we deemed these data ready
for analysis.

\hypertarget{gender}{%
\subparagraph{Gender}\label{gender}}

The Zeitgeist included six multiple-choice options for
gender-identification: Cisgender Female, Cisgender Male, Non-Binary,
Transgender Male, Transgender Female and These Options Don't Define Me.
Administration only had two categories: Female, Male. We chose to
consider only the following statistics: Cisgender Female
(Zeitgeist)::Female (Admin) and Cisgender Male (Zeitgeist)::Male
(Admin). This decision was made to uphold statistical rigor and avoid
mis-gendering Zeitgeist respondents. We regret that folks who selected
genders other than Cisgender Female and Cisgender Male were forced to
select Female or Male in their College Applications.

\#\#\#\#\#Race For the Race section of the Zeitgeist, students were
given eight options from which they were free to choose as many as they
wished. These options were: ``White'', ``Asian', ``Hispanic or Latino
Origin'', ``Black or African American'', ``Middle Eastern or North
African'', ``American Indian or Alaskan Native'' and ``Native Hawaiian
or other Pacific Islander''. There were 1267 responses, of 1134
respondents, meaning at maximum, 153 respondents selected more than one
race. From our data, we have no way to determine what selections these
153 respondents made, meaning the data has unexplainable variance. The
Middlebury Admin data only allows a single selection. It does not
include a category for ``Middle Eastern or North African'' and includes
categories for ``International'', ``Race and/or ethnicity unknown'', and
``Two or more races''. Rather than make assumptions about
respondents'/students' race, we only selected features from datasets
that matched one-to-one. As we have no way to remove the variance from
respondents who selected multiple races, we must inspect results from
permutation testing on this data with extreme scrutiny.

\#\#\#\#\#Geographic Location We did not receive the number of
international students enrolled during the spring of 2022; therefore, we
are excluding international students from our statistical analysis.

\#\#\#Phase 2 Testing If we see any significant differences, we want to
change our assumptions and run the test again to ensure that we are not
making any false claims based on false assumptions. In this section, I
will outline our changes in assumptions for phase 2. Note that this
section was written prior to doing any statistical tests on our data.

Like before, our Class Year, Geographic Location, and Academic Major
metrics matched one-to-one after removing the ``I prefer not to answer''
category, so we were not making any significant assumptions and will not
run phase 2 testing for these categories.

\#\#\#\#\#Gender We know that Zeitgeist surveyed respondents with six
multiple-choice options for gender identification. Middlebury, however,
only had two options two fill in: ``Female'' and ``Male''. For our phase
1 testing, we just ignored the Zeitgeist responses that were neither
``Female'' nor ``Male''. In phase 2 testing, we will make the assumption
that people in other gender categories were forced to respond either
``Female'' or ``Male''. We will assume that such a respondent's choice
of ``Female'' or ``Male'' was random with probability 0.5 of choosing
either category. Our reasoning is that when non-binary people apply to
Middlebury and have to select one of two categories which don't apply to
them, they will pick randomly. After randomly re-assigning gender
non-binary people, we will run our statistical test again.

\#\#\#\#\#Race For phase 1 testing, we made some big assumptions about
how we should test our data (see data modification above). For phase 2,
we want to try a different approach. We will assume that those in
administration's data who selected ``Two or more races'' would have
selected exactly two races if they were given the Zeitgeist survey. We
will randomly assign each ``Two or more races'' selection in the
administration data to two of the race categories with according to the
distribution at Middlebury given by the admin data (excluding the two or
more races category). Here we are adjusting our assumptions about the
population rather than the Zeitgeist sample. After our changes we will
run the test again with our new data.

\#\#\#\#\#Major At Middlebury, students often don't declare their major
until the end of their second year---even after they know what they want
to study. In phase 2 we will assume that some students marked down their
intended major in the Zeitgeist survey even though they were still
``undeclared'' according to Middlebury's records. To account for this,
we will remove the ``undeclared'' category from both data sets---making
the assumption that people who gave their intended major on the
Zeitgeist survey, while being undeclared, are not distributed in any
special way. That is, they follow the distributions of majors already
present in both the Middlebury and Zeitgeist data.

\#Results

\#\#\#Class When looking at our class category, we observe a \(p\)-value
of about 0.17437, meaning that we have about a 17\% chance of observing
something more extreme than the test statistic of our sample given that
the proportion of students in each class is randomly sampled from the
Middlebury population. See \textbf{Exhibit B} in the appendix for the
\texttt{R} code. This is not rare enough to reject the null hypothesis
because the \(p\)-value is larger than \(\alpha^*\)

\#\#\#Geography For our geography category, we get a \(p\)-value of
about 0.04467 meaning there is about a 4\% chance of observing something
more rare than our sample's test statistic given that Zeitgeist is
randomly sampled from the administration's population data
(\textbf{Exhibit C}). This is still larger than \(\alpha^*\), so we do
not have enough evidence to reject the null hypothesis.

\begin{enumerate}
\def\labelenumi{\arabic{enumi})}
\setcounter{enumi}{3}
\tightlist
\item
  Conclusion: Explain how your research question was answered and what
  possible meaningful takeaway you/the reader should have.
\end{enumerate}

Thank Yous:

Consult for ethical data modification Sophie Hochman, Joint GSFS
Sociology Major.

\end{document}
